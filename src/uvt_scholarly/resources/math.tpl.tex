% SPDX-FileCopyrightText: 2026 Alexandru Fikl <alexfikl@gmail.com>
% SPDX-License-Identifier: CC-BY-SA-4.0

\documentclass[11pt, a4paper]{article}

% {{{ BEGIN packages

% NOTE: toggle the following lines to switch to Romanian
\usepackage[romanian]{babel}
\selectlanguage{romanian}

\usepackage{geometry}
\geometry{top=2cm, left=1cm, right=1cm, bottom=2cm}

\usepackage[table]{xcolor}

\definecolor{UrlColor}{HTML}{0072BD}
\usepackage[hypertexnames=false,breaklinks=true]{hyperref}
\hypersetup{%
    colorlinks=true,
    urlcolor=UrlColor,
    citecolor=UrlColor,
    linkcolor=UrlColor
}
\renewcommand{\UrlFont}{\bfseries\ttfamily}

\usepackage[shortlabels]{enumitem}

\usepackage{tabularray}
\UseTblrLibrary{longtblr}
\SetTblrTemplate{caption}{empty}
\SetTblrTemplate{capcont}{empty}

\DeclareTblrTemplate{contfoot-text}{normal}{Continuat pe pagina următoare}
\SetTblrTemplate{contfoot-text}{normal}

% }}} END

% {{{ BEGIN formatting


% set font
\ifpdftex
    \usepackage[sc]{mathpazo}
\else
    % NOTE: use a more old timey Times New Roman lookalike
    \usepackage{unicode-math}
    \setmainfont{TeX Gyre Termes}[FakeBold=1]
    \setmathfont{Stix Two Math}[FakeBold=1]
    \setmonofont{Inconsolata}[Scale=MatchLowercase]
\fi

% remove paragraph indentation
\setlength{\parindent}{0pt}

% remove page numbers
\pagestyle{empty}

% define some standard colors used in the official templates
\definecolor{TableGray}{RGB}{242, 242, 242}
\definecolor{TableYellow}{RGB}{255, 255, 153}

% }}} END

% {{{ BEGIN commands

\NewDocumentCommand \JinjaBlock {+m} {}
\NewDocumentCommand \JinjaVar {+m} {}

% }}}

\begin{document}

\begin{center}
{\Huge \textbf{Fișă de verificare: \JinjaVar{candidate.qualname}}}
\bigskip

{\Large Standarde minimale pentru conferirea titlului de \JinjaVar{criteria.position_name}}

{\Large Departament: Matematică}
\end{center}

\vspace{2cm}

\textbf{Indicatori precizați în Anexa 1 la ORDINUL nr. 6129 din 20.12.2016, publicată în
Monitorul Oficial,  Partea I, nr. 123bis/15.02.2017.}

\bigskip

{
\renewcommand{\arraystretch}{1.5}

\begin{tblr}{width=\textwidth, colspec={|l|X|X|}, hlines, vlines}
\SetRow{bg=TableYellow}
\textbf{Indicator}
& \textbf{Standard pentru \JinjaVar{criteria.position_name}}
& \textbf{Valoarea realizată a indicatorului} \\
S &
$\ge \JinjaVar{criteria.min_ris}$ &
\JinjaVar{'{:.3f}'.format(candidate.ris)} \\
S\textsubscript{recent} &
$\ge \JinjaVar{criteria.min_recent_ris}$ &
\JinjaVar{'{:.3f}'.format(candidate.recent_ris)} \\
C &
$\ge \JinjaVar{criteria.min_citations}$ &
\JinjaVar{candidate.total_citations} \\
\end{tblr}
}

{\footnotesize
\begin{itemize}[leftmargin=*]
    \item S: suma scorurilor relative de influență (maximul peste ultimii 5 ani) a
    tuturor articolelor publicate (cu un scor de peste 0.5).
    \item S\textsubscript{recent}: suma scorurilor relative de influență (maximul
    peste ultimii 5 ani) a tuturor articolelor publicate \textbf{în ultimii 7 ani}
    (cu un scor de peste 0.5).
    \item C: numărul de citări ale articolelor luate în considerare pentru indicatorul S.
    Nu se iau în considerare articole care au ca autor candidatul.
\end{itemize}
}
\vfill

{\large
\noindent \textbf{Timișoara, \today \hfill \JinjaVar{candidate.qualname}}
}

\vspace{1.5cm}

\clearpage

{
\textbf{Universitatea de Vest din Timișoara}

\textbf{Facultatea de Matematică}

\textbf{\JinjaVar{candidate.qualname}}
}

\bigskip
\begin{center}
{\Large \textbf{Calculul coeficienților S și S\textsubscript{recent}}}
\end{center}
\bigskip

{\footnotesize

\begin{longtblr}{
    colspec={|Q[wd=1.5em,c]
             |X
             |Q[wd=4em,c]
             |Q[wd=2.9em,c]
             |Q[wd=2.7em,c]
             |Q[wd=1.8em,c]|},
    hlines,
    vlines,
    long}
\SetRow{valign=m, bg=TableYellow}
\textbf{Nr. crt.} &
\textbf{Articol -- referință bibliografică} &
\textbf{Publicat în ultimii 7 ani} &
\textbf{SRI revistă} &
\textbf{Nr. autori} &
\textbf{Scor} \\
\JinjaBlock{for pub in candidate.publications}
\JinjaVar{loop.index} &
\JinjaVar{pub | format_pub} &
\JinjaVar{pub | is_recent} &
\JinjaVar{'{:.3f}'.format(pub | get_score('RIS'))} &
\JinjaVar{pub.authors | length} &
\JinjaVar{'{:.3f}'.format(pub | get_average_score('RIS'))} \\
\JinjaBlock{endfor}
\SetRow{bg=TableGray} &
\textbf{Total} &
\SetCell[c=2]{l} \textbf{S = \JinjaVar{candidate.ris}} & &
\SetCell[c=2]{l} \textbf{S\textsubscript{recent} = \JinjaVar{candidate.recent_ris}} \\
\end{longtblr}
}

{\footnotesize
Scor: SRI revistă / Nr. Autori. Pentru poziția de Lector (sau Cercetător
Științific III) nu se împarte la numărul de autori.
}

\clearpage

\begin{center}
{\Large \textbf{Citări (în reviste S\textsubscript{i} $\ge$ 0.5)}}
\end{center}
\bigskip

{\footnotesize

\begin{longtblr}{
    width=\textwidth,
    colspec={|Q[wd=1.5em,c]
             |X
             |X
             |Q[wd=3.3em,c]|},
    hlines,
    vlines}
\SetRow{valign=m, bg=TableYellow}
\textbf{Nr. crt.} &
\textbf{Articolul citat} &
\textbf{Revista și articolul în care a fost citat} &
\textbf{SRI Revistă} \\
\JinjaBlock{set counter = namespace(value=0)}
\JinjaBlock{for pub in candidate.publications}
    \JinjaBlock{-for cite in pub.cited_by}
        \JinjaBlock{set counter.value = counter.value + 1}
\JinjaVar{counter.value} &
\JinjaBlock{-if loop.index == 1} \JinjaVar{pub | format_pub} \JinjaBlock{endif-} &
\JinjaVar{cite | format_pub} &
\JinjaVar{'{:.3f}'.format(cite | get_score('RIS'))} \\
    \JinjaBlock{-endfor}
\JinjaBlock{-endfor}
\SetRow{bg=TableGray} &
\textbf{Total} &
\SetCell[c=2]{r} \textbf{C = \JinjaVar{candidate.total_citations}}
\end{longtblr}
}
\end{document}

% vim:foldmarker=<<<,>>>:foldmethod=marker:spell spelllang=ro
% kate: default-dictionary ro_RO;
